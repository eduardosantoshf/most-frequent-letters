%
% General structure for the revdetua class:
%
\documentclass[...]{revdetua}
\usepackage{graphicx}
\usepackage{float}

%%%%%%%% Python code snippet %%%%%%%%

% Custom colors
\usepackage{color}
\definecolor{deepblue}{rgb}{0,0,0.5}
\definecolor{deepred}{rgb}{0.6,0,0}
\definecolor{deepgreen}{rgb}{0,0.5,0}

\usepackage{listings}

% Python style for highlighting
\newcommand\pythonstyle{\lstset{
language=Python,
basicstyle=\ttm\footnotesize, %added \footnotesize to /basicstyle to make the correcy code siz
morekeywords={self},              % Add keywords here
keywordstyle=\ttb\color{deepblue},
emph={MyClass,__init__},          % Custom highlighting
emphstyle=\ttb\color{deepred},    % Custom highlighting style
stringstyle=\color{deepgreen},
%frame=tb,                         % Any extra options here
showstringspaces=false
}}


% Python environment
\lstnewenvironment{python}[1][]
{
\pythonstyle
\lstset{#1}
}
{}

% listings to highlight code
\usepackage{pythonhighlight}

%%%%%%%%%%%%%%%%%%%%%%%%%%%%%%%%%%%%

%
% Valid options are:
%
%   longpaper --------- \part and \tableofcontents defined
%   shortpaper -------- \part and \tableofcontents not defined (default)
%
%   english ----------- main language is English (default)
%   portugues --------- main language is Portuguese
%
%   draft ------------- draft version
%   final ------------- final version (default)
%
%   times ------------- use times (postscript) fonts for text
%
%   mirror ------------ prints a mirror image of the paper (with dvips)
%
%   visiblelabels ----- \SL, \SN, \SP, \EL, \EN, etc. defined
%   invisiblelabels --- \SL, \SN, \SP, \EL, \EN, etc. not defined (default)
%
% Note: the final version should use the times fonts
% Note: the really final version should also use the mirror option
%

\begin{document}

\Header{1}{25}{janeiro}{2023}{0}
% Note: the month must be in Portuguese

\title{Most Frequent Letters}
\author{Eduardo Santos, nºmec 93107, eduardosantoshf@ua.pt} % or \author{... \and ...}
\maketitle

\begin{abstract}
The objective of this assignment was to identify the most frequent letters in text files  using different methods and to evaluate the quality of estimates regarding the exact counts. Three types of counters were implemented: an Exact Counter, a Decreasing Probability Counter, and a Frequent Counter. 
\end{abstract}

\begin{keywords}{Counting Exact Counter, Decreasing Probability Counter, Frequent Counter}
\end{keywords}

\section{Introduction}

\subsection{Decreasing Probability Counter}

The aim of a probabilistic counter is to count very large numbers using only a little space to store the counter. Counting a very large number of events using an exact counter will result in a large memory usage, which is something that should be avoided, as memory is expensive, and makes a program less efficient. To mitigate this problem, probabilistic counters were created. So, for each call to an increment method on the counter, its actual value is updated with probability \textit{p}. Using this method, we are trading accuracy for the ability to count up to very large numbers with little storage space.

Even though nowadays memory is no longer scarce, this approach is still useful when treating/counting massive data volumes, when there is a need for quick and memory-efficient processing.

\subsection{Frequent Counter}

A streaming algorithm is an algorithm for processing data streams in which the input is presented as a sequence of items and can be examined in only a few passes, typically just one.
The frequent problem happens when, given a sequence of items, we want to identify those which occur most frequently. This can also be expressed as finding all items whose frequency exceeds a specified fraction of the total number of items.

\section{Problem Description}

The  goal of this assignment was to identify the most frequent letters in text files using different methods and to evaluate the quality of estimates regarding the exact counts.

In order to accomplish that, three different approaches were developed and tested:
\begin{itemize}
    \item exact counter
    \item approximate counter - decreasing probability counter with probability \(\frac{1}{\sqrt{2}^k}\)
    \item frequent counter - Misra \& Gries algorithm to identify frequent items in data streams
\end{itemize} 

The testing involved an analysis of the computational efficiency and limitations of the developed approaches was carried out, in terms of absolute and relative errors, computing the mean, minimum, and maximum of each error, as well as computing the standard deviation and variance.
Finally, for each method, the most frequent letters were identified, checking if they were in the same relative order.

\section{Implementation Description}

Running the \textbf{main.py}, with the \textbf{--help} flag, a few running options are presented.

\begin{figure}[!htb]
    \centering
    \includegraphics[width=1\columnwidth]{./figures/usage}
    \caption{Help Menu of the Main Program}
    \label{fig: Help Menu}
\end{figure}

The \textbf{-t} flag represents the filename of the literary work to be read and processed by the program, which can be found inside the \textbf{/texts/} directory. The \textbf{-s} flag represents the filename of the stop-words to be ignored when processing the text file. The stop-words files can be found inside the \textbf{/stop-words/} directory. After the previous parameters, we have to specify the counter type to be computed, using one of the following:
\begin{itemize}
    \item \textbf{exact} - count using the exact counter
    \item \textbf{decreasing} - count using the decreasing probability counter (with probability \(\frac{1}{\sqrt{2}^k}\))
    \item \textbf{frequent} - count using the frequent counter (we must also specify the \textit{k} parameter, using the flag \textbf{-k})
\end{itemize}

The \textbf{counter.py} contains the main logic of the solution, using the \textbf{ABC}\cite{abstract_base_classes} (Abstract Base Classes) Python module, the \textbf{ExactCounter}, \textbf{DecreasingProbabilityCounter}, and \textbf{FrequentCounter} classes extend the \textbf{Counter} parent class, using an OOP (object-oriented programming) model.
Inside the \textbf{Counter} class there is the \textit{read\_letters()} method, which is responsible for parsing the text file, removing the Project Gutenberg file headers and footers, all stop-words and punctuation marks, as well as converting all letters to uppercase.

\begin{python}[linenos, tabsize=1, breaklines]
def read_letters(self):
    with open(self.filename, 'r') as file:
        while True:
            line = file.readline()

            # ignore the Project Gutenberg's file headers
                if line.strip() in [header.value for header in Headers]: break
                
        while line:
            line = file.readline()

            # ignore the Project Gutenberg's file footers
            if line.strip() in [footer.value for footer in Footers]: break
            
            for words in line.split():
                # remove all stop-words and punctuation marks
                for word in regex.findall('\p{alpha}+', words):
                    for letter in word:
                        self.parsed_letters.append(letter.upper())
\end{python}

\subsection{Exact Counter}

Inside the \textbf{ExactCounter} class we can find the \textit{compute()} method, which is responsible for counting the exact number of occurrences of each letter from the literary work.

\subsection{Decreasing Probability Counter}

The \textbf{DecreasingProbabilityCounter} class implements the Decreasing Probability Counter as a probabilistic counter, with the increment being made with probability \(\frac{1}{\sqrt{2}^k}\), where \textit{k} represents the number of occurrences of each letter. If the counter has value \textit{k}, the algorithm increases the number of occurrences of the letter with the previously mentioned probability. Due to \textit{k} and the probability being inversely proportional, as \textit{k} increases, the probability of incrementing the counter will be much smaller. This method allows the counting of a large number of events using a small amount of memory.
The estimated value of the counter
for each letter can be calculated using the following formula:

\[
\frac{\sqrt{2}^k - \sqrt{2} + 1}{\sqrt{2} - 1}
\]

\subsection{Frequent Counter}

The \textbf{FrequentCount} class implements a Frequent Counter as a Data Stream Algorithm. The goal is to establish an estimate for the frequency of any stream letter. The frequent count algorithm implemented was the Misra & Gries algorithm. This uses a parameter \textit{k} that controls the quality of the results given. It uses a \textit{{letter: counter}} dictionary, with at most \textit{(k - 1)} counters, at any time.
This algorithm provides, for any letter, \textit{l}, a frequency estimate \textit{f_{l}^*} satisfying

\[
f_l - \frac{m}{k} \leq f_{l}^* \leq f_l
\]

were \textit{m} is the length of the data stream or, in this case, the total number of letters in the text. If some letter has \(f_l > \frac{m}{k}\), its counter \textit{A[l]} will be positive, i.e., no item with frequency \(\frac{m}{k}\) is missed.

\section{Results and Discussion}


\section{Conclusion}


\begin{thebibliography}{9}


% https://en.wikipedia.org/wiki/Approximate_counting_algorithm
% https://www.algorithm-archive.org/contents/approximate_counting/approximate_counting.html
% https://people.csail.mit.edu/rrw/6.045-2020/encalgs-mg.pdf
% https://medium.com/nerd-for-tech/the-streaming-model-and-how-to-estimate-the-most-frequent-elements-with-the-misra-gries-algorithm-c880bbe7218b
% https://en.wikipedia.org/wiki/Streaming_algorithm#:~:text=In%20computer%20science%2C%20streaming%20algorithms,passes%20(typically%20just%20one).
% https://github.com/jesuszarate/FrequentItems/blob/master/MisraGries/misra-gries.py
% https://github.com/joshuaeitan/misra_gries/blob/master/misra_gries/misra_gries.py
% 

\bibitem{regular_expressions}
Python Software Foundation. (Dec 22, 2022). re — Regular expression operations. \url{https://docs.python.org/3/library/re.html}

\bibitem{abstract_base_classes}
Python Software Foundation. (Dec 22, 2022). abc — Abstract Base Classes. \url{https://docs.python.org/3/library/abc.html}

\end{thebibliography}


% use a field named url or \url{} for URLs
% Note: the \bibliographystyle is set automatically

\end{document}
